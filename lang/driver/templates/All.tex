\documentclass[12pt]{scrartcl}
\usepackage[utf8]{inputenc}
\usepackage{alltt}
\usepackage{xcolor}
% Color definitions
\definecolor{polBlack}{rgb}{0,0,0}
\definecolor{polBlue}{rgb}{0.06, 0.2, 0.65}
\definecolor{polGreen}{RGB}{0,155,85}
\definecolor{polRed}{rgb}{0.8,0.4,0.3}
\definecolor{polCyan}{rgb}{0.0, 1.0, 1.0}
\definecolor{polMagenta}{rgb}{0.8, 0.13, 0.13}
\definecolor{polYellow}{rgb}{0.91, 0.84, 0.42}
\definecolor{polWhite}{rgb}{1,1,1}

\addtokomafont{title}{\raggedright}
\date{}

\newcommand*{\setTT}[1]{\texttt{#1}}
\newcommand{\polType}[1]{\textcolor{polRed}{\setTT{#1}}}
\newcommand{\polCtor}[1]{\textcolor{polBlue}{\setTT{#1}}}
\newcommand{\polDtor}[1]{\textcolor{polGreen}{\setTT{#1}}}
\newcommand{\polKw}[1]{\textcolor{polMagenta}{\setTT{#1}}}
\newcommand{\polVar}[1]{\setTT{#1}}
\newcommand{\polComment}[1]{\textcolor{polCyan}{\setTT{#1}}}
\newcommand*{\polText}[1]{\setTT{#1}}

\title{<< name >>}

\begin{document}

\maketitle

% Program
\section*{Initial Program}
The code written by the programmer:
\input{./<< name >>.tex}

\section*{Sequent Calculus}
A $\lambda\mu\tilde\mu$-calculus based intermediate language.

\subsection*{Compiled}
The result of compiling the original program.
This translation avoids generating administrative redexes.

\input{../compiled/<< name >>.tex}

\subsection*{Focused}
The result of focusing. After this step, every constructor, destructor and call is only applied to variables.

\input{../focused/<< name >>.tex}

\section*{AxCut}
AxCut is an intermediate language that exploits the symmetry of the classical sequent calculus. It corresponds to a one-sided presentation of the sequent calculus.

\subsection*{Shrunk}
The result of translating the focused representation into AxCut.

\input{../shrunk/<< name >>.tex}

\subsection*{Linearized}
The result of making contraction and weakening explicit.

\input{../linearized/<< name >>.tex}

\section*{Assembly}
The generated assembly from the code generator.
\input{../assembly/<< backend >>/<< name >>.tex}

\end{document}
