\documentclass[nonacm]{acmart}
\settopmatter{printfolios=true,printccs=true,printacmref=true}

%%
%% Bibliography and Citation Style
%%
\bibliographystyle{ACM-Reference-Format}
\citestyle{acmauthoryear}

\usepackage{mathtools}
\usepackage{amsmath}
\usepackage{cleveref}
\usepackage{bussproofs}
\usepackage{listings}
\usepackage{csquotes}
\usepackage{tikz}

%%
%% Macros
%%
%%
%% Terms
%%
\newcommand{\lit}[1]{\ensuremath{\ulcorner #1 \urcorner}}
\newcommand{\cut}[2]{\ensuremath{\langle #1 \mid #2 \rangle}}
\newcommand{\done}{\ensuremath{\mathbf{done}}}
\newcommand{\ifz}[3]{\ensuremath{\mathbf{ifz}(#1, #2, #3)}}
\newcommand{\letin}[3]{\ensuremath{\mathbf{let}\ #1 = #2\ \mathbf{in}\ #3}}
\newcommand{\caseof}[2]{\ensuremath{\mathbf{case}\ #1\ \mathbf{of}\ \{ #2 \}}}
\newcommand{\case}[1]{\ensuremath{\mathbf{case}\ \{ #1 \}}}
\newcommand{\cocase}[1]{\ensuremath{\mathbf{cocase}\ \{ #1 \}}}
\newcommand{\goto}[2]{\ensuremath{\mathbf{goto}(#1; #2)}}
\newcommand{\lab}[2]{\ensuremath{\mathbf{label}\ #1\ \{#2 \}}}
\newcommand{\defi}[2]{\ensuremath{\mathbf{def}\ #1 \coloneq #2}}

%%
%% AxCut Terms
%%
\newcommand{\jump}[1]{\mathbf{jump}\, #1}
\newcommand{\invoke}[2]{\mathbf{invoke}\, #1\, #2}
\newcommand{\switch}[2]{\mathbf{switch}\, #1\, #2}
\newcommand{\substitute}[2]{\mathbf{substitute}[#1];#2}


%%
%% Languages and Calculi
%%
\newcommand{\lambdamumu}{$\lambda\mu\tilde\mu$-calculus}
\newcommand{\surfacelang}{\textbf{Fun}}
\newcommand{\targetlang}{\textbf{Core}}
\newcommand{\machinelang}{\textbf{AxCut}}

%%
%% Types / Typing Rules / Typing Contexts
%%
\newcommand{\tyint}{\mathbf{Int}}
\newcommand{\wrap}[1]{:^{\text{\tiny#1}}}
\newcommand{\prd}{\wrap{prd}}
\newcommand{\cnt}{\wrap{cns}}
\newcommand{\wellformed}[1]{#1\ \textsc{Ok}}

%%
%% Evaluation / Reduction / Operational Semantics
%%
\newcommand{\reducesto}{\ensuremath{\triangleright}}


%%
%% Translations / Transfomations
%%
\newcommand{\focus}[1]{\ensuremath{\mathcal{F}(#1)}}
\newcommand{\name}[1]{\ensuremath{\mathcal{N}(#1)}}
\newcommand{\bind}[2]{\ensuremath{\mathsf{bind}(#1)[#2]}}
\newcommand{\fresh}[1]{\ensuremath{(#1\text{ fresh})}}
\newcommand{\translate}[1]{\ensuremath{\llbracket #1 \rrbracket}}
\newcommand{\translatestar}[2]{\ensuremath{\llbracket #1 \rrbracket_{#2}^{\ast}}}


\begin{document}
%%
%% Title information
%%
\title{Compiling with the Sequent Calculus}
\subtitle{A Unifying Approach
for Control Effects and Codata Types}

%%
%% Keywords
%%
\keywords{Intermediate representations, continuations, codata types, control effects}

%%
%% CCS Classification
%%

\begin{CCSXML}
  <ccs2012>
     <concept>
         <concept_id>10003752.10003753.10003754.10003733</concept_id>
         <concept_desc>Theory of computation~Lambda calculus</concept_desc>
         <concept_significance>500</concept_significance>
         </concept>
     <concept>
         <concept_id>10011007.10011006.10011041</concept_id>
         <concept_desc>Software and its engineering~Compilers</concept_desc>
         <concept_significance>500</concept_significance>
         </concept>
     <concept>
         <concept_id>10011007.10011006.10011008.10011024.10011027</concept_id>
         <concept_desc>Software and its engineering~Control structures</concept_desc>
         <concept_significance>300</concept_significance>
         </concept>
   </ccs2012>
\end{CCSXML}

\ccsdesc[500]{Theory of computation~Lambda calculus}
\ccsdesc[500]{Software and its engineering~Compilers}
\ccsdesc[300]{Software and its engineering~Control structures}

%%
%% Author: David Binder
%%
\author{David Binder}
\orcid{0000-0003-1272-0972}
\affiliation{
  \department{Department of Computer Science}
  \institution{University of Tübingen}
  \city{Tübingen}
  \country{Germany}
}
\email{david.binder@uni-tuebingen.de}

%%
%% Author: Marco Tzschentke
%%
\author{Marco Tzschentke}
\orcid{0009-0004-8834-2984}
\affiliation{
  \department{Department of Computer Science}
  \institution{University of Tübingen}
  \city{Tübingen}
  \country{Germany}
}
\email{marco.tzschentke@uni-tuebingen.de}

%%
%% Author: Marius Müller
%%
\author{Marius Müller}
\orcid{0000-0002-0260-6298}
\affiliation{
  \department{Department of Computer Science}
  \institution{University of Tübingen}
  \city{Tübingen}
  \country{Germany}
}
\email{mari.mueller@uni-tuebingen.de}

%%
%% Author: Philipp Schuster
%%
\author{Philipp Schuster}
\orcid{0000-0001-8011-0506}
\affiliation{
  \department{Department of Computer Science}
  \institution{University of Tübingen}
  \city{Tübingen}
  \country{Germany}
}
\email{philipp.schuster@uni-tuebingen.de}

%%
%% Author: Jonathan Immanuel Brachthäuser
%%
\author{Jonathan Immanuel Brachthäuser}
\orcid{0000-0001-9128-0391}
\affiliation{
  \department{Department of Computer Science}
  \institution{University of Tübingen}
  \city{Tübingen}
  \country{Germany}
}
\email{jonathan.brachthaeuser@uni-tuebingen.de}

%%
%% Author: Klaus Ostermann
%%
\author{Klaus Ostermann}
\orcid{0000-0001-5294-5506}
\affiliation{
  \department{Department of Computer Science}
  \institution{University of Tübingen}
  \city{Tübingen}
  \country{Germany}
}
\email{klaus.ostermann@uni-tuebingen.de}

%%
%% Abstract
%%
\begin{abstract}
  Compiling a high-level functional programming language to machine code that can be executed efficiently on a modern machine is a complicated task, since we have to traverse many different levels of abstraction.
  This is particularly challenging if the language contains some form of control effects and a mix of different calling conventions, such as call-by-value data types and call-by-name codata types.
  In this paper we tell the complete story, starting from a simple functional programming language with control effects and both data and codata types, and ending up with Risc-V and x86-64 machine code.
  The novelty of our approach lies in the fact that we use the sequent calculus, and sequent calculus inspired languages, in specifying the intermediate stages of our compiler.
  In that sense, we view this work as a continuation, and generalization, of
  Andrew Appel's landmark work on \enquote{Compiling with Continuations}.
\end{abstract}

\maketitle

%%
%% Section: Introduction
%%
\section{Introduction}
\label{sec:introduction}
This article is structured as follows:
\begin{itemize}
    \item In \cref{sec:fun} we introduce the surface language \surfacelang.
    \item In \cref{sec:core} we introduce the intermediate language \targetlang.
    \item In \cref{sec:translation} we show how to translate programs from the surface language \surfacelang\ to the intermediate language \targetlang.
    \item In \cref{sec:naming-transformation} we show a transformation on programs in \targetlang\ which names all intermediate computations, similar to ANF or focusing transformations.
\end{itemize}


%%
%% Section: The Surface Language Fun
%%
\section{The Surface Language Fun}
\label{sec:fun}
\begin{definition}[Syntactic Conventions]
  We use the following metavariables for all languages:
  \[
    \begin{array}{rcll}
      \odot  & \coloneqq & + \mid - \mid * & \emph{Arithmetic Operators}
    \end{array}
  \]
\end{definition}

\begin{definition}[Syntax of Fun]
  We assume a finite set of names $\mathcal{N}$ containing type names $T\in\mathcal{N}$, constructor names $K\in\mathcal{N}$ and destructor names $D\in\mathcal{N}$.
  \[ 
    \begin{array}{r c l l}
      t & \coloneqq & x \mid \lit{n} \mid t \odot t \mid \ifz{t}{t}{t} \mid \letin{x}{t}{t} \mid f(\overline{t}; \overline{\alpha}) & \emph{Terms}\\
      & \mid & K(\overline{t}) \mid \caseof{t}{\overline{K(\overline{x}) \Rightarrow t}} \mid t.D(\overline{t}) \mid \cocase{\overline{D(\overline{x}) \Rightarrow t}} & \\
      & \mid & \lab{\alpha}{t} \mid \goto{t}{\alpha} & \\
      \tau & \coloneqq & \tyint \mid T & \emph{Types} \\
      \kappa & \coloneq & \mathbf{data} \mid \mathbf{codata} \mid \mathbf{prim} & \emph{Kinds} \\
      \Gamma & \Coloneqq & \emptyset \mid \Gamma, x \prd \tau \mid \Gamma, \alpha \cnt \tau & \emph{Typing Contexts} \\
      \delta & \coloneqq & \mathbf{data}\ T\ \{ \overline{K(\overline{x:\tau})} \} & \emph{Data Type Definition}\\
       & \mid & \mathbf{codata}\ T\ \{ \overline{D(\overline{x:\tau}) : \tau}\} & \emph{Codata Type Definition}\\
       & \mid & \defi{f(\overline{x};\overline{\alpha})}{t} & \emph{Top-Level Definitions}\\
      \Theta & \coloneqq & \overline{\delta} & \emph{Programs}\\
    \end{array}
  \]
\end{definition}

%%
%% Subsec: Typing Rules
%%
\subsection{Typing Rules}
\label{subsec:fun:typing-rules}


\begin{minipage}{\textwidth}
\begin{minipage}{0.3\textwidth}
  \begin{prooftree}
    \AxiomC{$x \prd \tau \in \Gamma$}
    \RightLabel{\textsc{Var}}
    \UnaryInfC{$\Theta\mid\Gamma \vdash x : \tau$}
  \end{prooftree}
\end{minipage}
\hfill
\begin{minipage}{0.3\textwidth}
  \begin{prooftree}
    \AxiomC{\phantom{$x : \tau \in \Gamma$}}
    \RightLabel{\textsc{Lit}}
    \UnaryInfC{$\Theta\mid\Gamma \vdash \lit{n}:\tyint$}
  \end{prooftree}
\end{minipage}
\hfill
\begin{minipage}{0.3\textwidth}
  \begin{prooftree}
    \AxiomC{$\Theta\mid\Gamma \vdash t_1: \tyint \quad \Theta\mid\Gamma \vdash t_2: \tyint$}
    \RightLabel{\textsc{Op}}
    \UnaryInfC{$\Theta\mid\Gamma \vdash t_1\odot t_2 : \tyint$}
  \end{prooftree}
\end{minipage}
\hfill
\vspace{1em}

\begin{minipage}{0.55\textwidth}
  \begin{prooftree}
    \AxiomC{$\Theta\mid\Gamma \vdash n : \tyint$}
    \AxiomC{$\Theta\mid\Gamma \vdash t_1 : \tau$}
    \AxiomC{$\Theta\mid\Gamma \vdash t_2 : \tau$}
    \RightLabel{\textsc{Ifz}}
    \TrinaryInfC{$\Theta\mid\Gamma \vdash \ifz{n}{t_1}{t_2} : \tau$}
  \end{prooftree}
\end{minipage}
\hfill
\begin{minipage}{0.4\textwidth}
  \begin{prooftree}
    \AxiomC{$\Theta\mid\Gamma \vdash t_1 : \tau_1$}
    \AxiomC{$\Theta\mid\Gamma, x \prd \tau_1 \vdash t_2 : \tau_2$}
    \RightLabel{\textsc{Let}}
    \BinaryInfC{$\Theta\mid\Gamma \vdash \letin{x}{t_1}{t_2} :\tau_2$}
  \end{prooftree}
\end{minipage}
\hfill
\vspace{1em}

\begin{minipage}{\textwidth}
  \begin{prooftree}
    \AxiomC{$\mathbf{def}\ f(\overline{x_i \prd \tau_i};\overline{\alpha_j\cnt\tau_j}) :\tau \in \Theta$}
    \AxiomC{$\overline{\Theta\mid\Gamma \vdash t_i : \tau_i}$}
    \AxiomC{$\overline{\beta_j \cnt \tau_j \in \Gamma}$}
    \RightLabel{\textsc{Call}}
    \TrinaryInfC{$\Theta\mid\Gamma \vdash f(\overline{t_i};\overline{\beta_j}) : \tau$}
  \end{prooftree}
\end{minipage}
\hfill
\vspace{1em}

\begin{minipage}{0.55\textwidth}
  \begin{prooftree}
    \AxiomC{$\Theta\mid\Gamma \vdash t : \tau$}
    \AxiomC{$\alpha \cnt \tau \in \Gamma$}
    \RightLabel{\textsc{Goto}}
    \BinaryInfC{$\Theta\mid\Gamma \vdash \goto{t}{\alpha} : \tau'$}
  \end{prooftree}
\end{minipage}
\begin{minipage}{0.4\textwidth}
  \begin{prooftree}
    \AxiomC{$\Theta\mid\Gamma, \alpha \cnt \tau \vdash t : \tau$}
    \RightLabel{\textsc{Label}}
    \UnaryInfC{$\Theta\mid\Gamma \vdash \lab{\alpha}{t} : \tau$}
  \end{prooftree}
\end{minipage}
\hfill
\vspace{1em}
\begin{minipage}{\textwidth}
  \begin{prooftree}
    \AxiomC{$\mathbf{data}\ T\ \{\ldots, K(\overline{x_i:\tau_i})\}\in\Theta$}
    \AxiomC{$\overline{\Theta\mid\Gamma \vdash t_i : \tau_i}$}
    \RightLabel{\textsc{Constructor}}
    \BinaryInfC{$\Theta\mid\Gamma \vdash K(\overline{t_i}) : T$}
  \end{prooftree}
\end{minipage}
\hfill
\vspace{1em}
\begin{minipage}{\textwidth}
  \begin{prooftree}
    \AxiomC{$\mathbf{data}\ T\{\overline{K_i(\overline{x_{i,j}:\tau_{i,j}}))}\}\in \Theta$}
    \AxiomC{$\overline{\Theta\mid\Gamma,\overline{x_{i,j}\prd\tau_{i,j}}\vdash t_i : \tau}$}
    \RightLabel{\textsc{Case}}
    \BinaryInfC{$\Theta\mid\Gamma\vdash \mathbf{case}\ t\ \mathbf{of}\ \{\overline{K_i(x_{i,j})\Rightarrow t_i}\} : \tau$}
  \end{prooftree}
\end{minipage}
\hfill
\vspace{1em}
\begin{minipage}{\textwidth}
  \begin{prooftree}
    \AxiomC{$\mathbf{codata}\ T\ \{\dots, D(\overline{x_i:\tau_i}):\tau\}\in \Theta$}
    \AxiomC{$\overline{\Theta\mid\Gamma\vdash t_i : \tau_i}$}
    \AxiomC{$\Theta\mid\Gamma \vdash t :T$}
    \RightLabel{\textsc{Destructor}}
    \TrinaryInfC{$\Theta\mid\Gamma \vdash t.D(\overline{t_i}) : \tau$}
  \end{prooftree}
\end{minipage}
\begin{minipage}{\textwidth}
  \begin{prooftree}
    \AxiomC{$\mathbf{codata}\ T \{\overline{D_i(\overline{x_{i,j}:\tau_{i,j}}):\tau_i}\}\in \Theta$}
    \AxiomC{$\overline{\Theta\mid\Gamma,\overline{y_{i,j}\prd\tau_{i,j}} \vdash t_i:\tau_i}$}
    \RightLabel{\textsc{Cocase}}
    \BinaryInfC{$\Theta\mid\Gamma \vdash \mathbf{cocase} \{ \overline{D_i(\overline{y_{i,j}})} \Rightarrow t_i \} : T$}
  \end{prooftree}
\end{minipage}
\end{minipage}
\vspace{1em}

\begin{minipage}{0.3\textwidth}
  \begin{prooftree}
    \AxiomC{}
    \RightLabel{\textsc{PrimKind}}
    \UnaryInfC{$\Theta \vdash \tyint : \mathbf{prim}$}
  \end{prooftree}  
\end{minipage}
\begin{minipage}{0.3\textwidth}
  \begin{prooftree}
    \AxiomC{$\mathbf{data}\ T\ \{\, \ldots \} \in \Theta$}
    \RightLabel{\textsc{DataKind}}
    \UnaryInfC{$\Theta \vdash T : \mathbf{data}$}
  \end{prooftree}  
\end{minipage}
\begin{minipage}{0.3\textwidth}
  \begin{prooftree}
    \AxiomC{$\mathbf{codata}\ T\ \{\, \ldots \} \in \Theta$}
    \RightLabel{\textsc{CodataKind}}
    \UnaryInfC{$\Theta \vdash T : \mathbf{codata}$}
  \end{prooftree}  
\end{minipage}









%%
%% Section: The Intermediate Language Core
%%
\section{The Intermediate Language Core}
\label{sec:core}
\begin{definition}[Syntax of Core]
    \[
      \begin{array}{rcll}
        p & \coloneqq & x \mid \lit{n} \mid \mu\alpha.s \mid K(\overline{p}; \overline{c}) \mid \cocase{\overline{D(\overline{x};\overline{\alpha}) \Rightarrow s}} & \emph{Producers}\\
        c & \coloneqq & \alpha \mid \tilde{\mu}x.s \mid D(\overline{p}; \overline{c}) \mid \case{\overline{K(\overline{x}; \overline{\alpha}) \Rightarrow s}}& \emph{Consumers}\\
        s & \coloneqq & \cut{p}{c} \mid \odot(p, p; c) \mid \ifz{p}{s}{s} \mid f(\overline{p}; \overline{c}) \mid \done & \emph{Statements}\\
        \Theta & \coloneqq & \emptyset \mid \defi{f(\overline{x}; \overline{\alpha})}{s}; \Theta & \emph{Programs}\\
      \end{array}
    \]
  \end{definition}

%%
%% Subsec: Typing Rules
%%
\subsection{Typing Rules}
\label{subsec:core:typing-rules}

  \begin{minipage}{0.21\textwidth}
    \begin{prooftree}
      \AxiomC{$\Gamma,\alpha\cnt \tau \vdash s$}
      \RightLabel{\textsc{$\mu$}}
      \UnaryInfC{$\Gamma \vdash \mu\alpha.s\prd\tau$}
    \end{prooftree}
  \end{minipage}
  \hfill
  \begin{minipage}{0.21\textwidth}
    \begin{prooftree}
      \AxiomC{$\Gamma,x\prd\tau \vdash s$}
      \RightLabel{\textsc{$\tilde{\mu}$}}
      \UnaryInfC{$\Gamma \vdash \tilde{\mu}x.s \cnt \tau$}
    \end{prooftree}
  \end{minipage}
  \hfill
   \begin{minipage}{0.21\textwidth}
    \begin{prooftree}
      \AxiomC{$x\prd\tau\in\Gamma$}
      \RightLabel{\textsc{Var$_1$}}
      \UnaryInfC{$\Gamma \vdash x\prd\tau$}
    \end{prooftree}
  \end{minipage}
  \hfill
  \begin{minipage}{0.21\textwidth}
    \begin{prooftree}
      \AxiomC{$\alpha\cnt\tau\in\Gamma$}
      \RightLabel{\textsc{Var$_2$}}
      \UnaryInfC{$\Gamma \vdash \alpha \cnt \tau$}
    \end{prooftree}
  \end{minipage}
  \hfill
  \vspace{1em}

  \begin{minipage}{0.45\textwidth}
    \begin{prooftree}
      \AxiomC{$\Gamma \vdash p \prd\tau$}
      \AxiomC{$\Gamma \vdash c \cnt \tau$}
      \RightLabel{\textsc{Cut}}
      \BinaryInfC{$\Gamma \vdash \cut{p}{c}$}
    \end{prooftree}
  \end{minipage}
  \hfill
  \begin{minipage}{0.45\textwidth}
    \begin{prooftree}
      \AxiomC{$\Gamma \vdash p \prd \tyint$}
      \AxiomC{$\Gamma \vdash s_1$}
      \AxiomC{$\Gamma \vdash s_2$}
      \RightLabel{\textsc{IfZ}}
      \TrinaryInfC{$\Gamma \vdash \ifz{p}{s_1}{s_2}$}
    \end{prooftree}
  \end{minipage}
  \hfill
  \vspace{1em}


  \begin{minipage}{0.35\textwidth}
    \begin{prooftree}
      \AxiomC{\quad}
      \RightLabel{\textsc{Lit}}
      \UnaryInfC{$\Gamma \vdash \lit{n} \prd \tyint$}
    \end{prooftree}
  \end{minipage}
  \hfill
  \begin{minipage}{0.6\textwidth}
    \begin{prooftree}
      \AxiomC{$\Gamma \vdash p_1 \prd \tyint$}
      \AxiomC{$\Gamma \vdash p_2 \prd \tyint$}
      \AxiomC{$\Gamma \vdash c \cnt \tyint$}
      \RightLabel{\textsc{binop}}
      \TrinaryInfC{$\Gamma \vdash \odot(p_1,p_2;c)$}
    \end{prooftree}
  \end{minipage}
  \hfill
  \vspace{1em}

  \begin{minipage}{\textwidth}
    \begin{prooftree}
      \AxiomC{$\mathbf{def}\ f(\overline{x_i \prd \tau_i};\overline{\alpha_j \cnt \tau_j}) \in P$}
      \AxiomC{$\overline{\Gamma \vdash p_i \prd \tau_i}$}
      \AxiomC{$\overline{\Gamma \vdash c_j \cnt \tau_j}$}
      \RightLabel{\textsc{Call}}
      \TrinaryInfC{$\Gamma \vdash f(\overline{p_i};\overline{c_j})$}
    \end{prooftree}
  \end{minipage}
  \hfill
  \vspace{1em}

  \begin{minipage}{\textwidth}
    \begin{prooftree}
      \AxiomC{$\Gamma \vdash s_1$}
      \AxiomC{$\Gamma, x\prd\tau,xs\prd\mathtt{List}(\tau)\vdash s_2$}
      \RightLabel{\textsc{Case-List}}
      \BinaryInfC{$\Gamma \vdash \case{\mathtt{Nil}\Rightarrow s_1,\mathtt{Cons}(x,xs) \Rightarrow s_2} \cnt \mathtt{List}(\tau)$}
    \end{prooftree}
  \end{minipage}
  \hfill
  \vspace{1em}

  \begin{minipage}{0.45\textwidth}
    \begin{prooftree}
      \AxiomC{\quad}
      \RightLabel{\textsc{Nil}}
      \UnaryInfC{$\Gamma \vdash \mathtt{Nil}\prd\mathtt{List}(\tau)$}
    \end{prooftree}
  \end{minipage}
  \hfill
  \begin{minipage}{0.45\textwidth}
    \begin{prooftree}
      \AxiomC{$\Gamma \vdash t_1\prd\tau$}
      \AxiomC{$\Gamma \vdash t_2\prd\mathtt{List}(\tau)$}
      \RightLabel{\textsc{Cons}}
      \BinaryInfC{$\Gamma \vdash \mathtt{Cons}(t_1,t_2)\prd\mathtt{List}(\tau)$}
    \end{prooftree}
  \end{minipage}
  \hfill
  \vspace{1em}

  \begin{minipage}{0.4\textwidth}
    \begin{prooftree}
      \AxiomC{$\Gamma \vdash t_1 \prd \tau_1$}
      \AxiomC{$\Gamma \vdash t_2 \prd \tau_2$}
      \RightLabel{\textsc{Tup}}
      \BinaryInfC{$\Gamma \vdash \mathtt{Tup}(t_1,t_2)\prd\mathtt{Pair}(\tau_1,\tau_2)$}
    \end{prooftree}
  \end{minipage}
  \hfill
  \begin{minipage}{0.55\textwidth}
    \begin{prooftree}
      \AxiomC{$\Gamma, x\prd\tau_1,y\prd\tau_2 \vdash s$}
      \RightLabel{\textsc{Case-Pair}}
      \UnaryInfC{$\Gamma \vdash \case{\mathtt{Tup}(x,y)\Rightarrow s}\cnt\mathtt{Pair}(\tau_1,\tau_2)$}
    \end{prooftree}
  \end{minipage}
  \hfill
  \vspace{1em}

  \begin{minipage}{0.45\textwidth}
    \begin{prooftree}
      \AxiomC{$\Gamma \vdash k\cnt\tau$}
      \RightLabel{\textsc{Hd}}
      \UnaryInfC{$\Gamma \vdash \mathtt{hd}(k)\cnt \mathtt{Stream}(\tau)$}
    \end{prooftree}
  \end{minipage}
  \hfill
  \begin{minipage}{0.45\textwidth}
    \begin{prooftree}
      \AxiomC{$\Gamma \vdash k\cnt\mathtt{Stream}(\tau)$}
      \RightLabel{\textsc{Tl}}
      \UnaryInfC{$\Gamma \vdash \mathtt{tl}(k)\cnt \mathtt{Stream}(\tau)$}
    \end{prooftree}
  \end{minipage}
  \hfill
  \vspace{1em}

  \begin{minipage}{\textwidth}
    \begin{prooftree}
      \AxiomC{$\Gamma,\alpha\cnt \tau \vdash s_1$}
      \AxiomC{$\Gamma,\beta\cnt\mathtt{Stream}(\tau)\vdash s_2$}
      \RightLabel{\textsc{Cocase-Stream}}
      \BinaryInfC{$\Gamma \vdash \cocase{\mathtt{hd}(\alpha)\Rightarrow s_1,\mathtt{tl}(\beta)\Rightarrow s_2} \prd \mathtt{Stream}(\tau)$}
    \end{prooftree}
  \end{minipage}
  \hfill
  \vspace{1em}

  \begin{minipage}{0.45\textwidth}
    \begin{prooftree}
      \AxiomC{$\Gamma \vdash k\cnt \tau_1$}
      \RightLabel{\textsc{Fst}}
      \UnaryInfC{$\Gamma \vdash \mathtt{fst}(k)\cnt\mathtt{LPair}(\tau_1,\tau_2)$}
    \end{prooftree}
  \end{minipage}
  \hfill
  \begin{minipage}{0.45\textwidth}
    \begin{prooftree}
      \AxiomC{$\Gamma \vdash k\cnt\tau_2$}
      \RightLabel{\textsc{Snd}}
      \UnaryInfC{$\Gamma \vdash \mathtt{snd}(k)\cnt \mathtt{LPair}(\tau_1,\tau_2)$}
    \end{prooftree}
  \end{minipage}
  \hfill
  \vspace{1em}

  \begin{minipage}{\textwidth}
    \begin{prooftree}
      \AxiomC{$\Gamma, \alpha \cnt\tau_1 \vdash s_1$}
      \AxiomC{$\Gamma, \beta \cnt\tau_2 \vdash s_2$}
      \RightLabel{\textsc{Cocase-LPair}}
      \BinaryInfC{$\Gamma \vdash \cocase{\mathtt{fst}(\alpha)\Rightarrow s_1, \mathtt{snd}(\beta)\Rightarrow s_2} \prd \mathtt{LPair}(\tau_1,\tau_2)$}
     \end{prooftree}
  \end{minipage}
  \hfill
  \vspace{1em}

  \begin{minipage}{0.4\textwidth}
    \begin{prooftree}
      \AxiomC{$\Gamma \vdash p\prd\sigma$}
      \AxiomC{$\Gamma \vdash c\cnt\tau$}
      \RightLabel{\textsc{Ap}}
      \BinaryInfC{$\Gamma \vdash \mathtt{ap}(p,c) \cnt \sigma\to\tau$}
    \end{prooftree}
  \end{minipage}
  \hfill
  \begin{minipage}{0.55\textwidth}
    \begin{prooftree}
      \AxiomC{$\Gamma, x\prd\sigma,\alpha\cnt\tau \vdash s$}
      \RightLabel{\textsc{Cocase-Fun}}
      \UnaryInfC{$\Gamma \vdash \cocase{\mathtt{ap}(x,\alpha) \Rightarrow s}\prd\sigma\to\tau$}
    \end{prooftree}
  \end{minipage}
  \hfill
  \vspace{1.5em}
  \rule{\textwidth}{0.4pt}
  \par
  \vspace{1em}
  \begin{minipage}{0.25\textwidth}
    \begin{prooftree}
      \AxiomC{}
      \RightLabel{\textsc{Wf-Empty}}
      \UnaryInfC{$\vdash \wellformed{\emptyset}$}
    \end{prooftree}
  \end{minipage}
  \hfill
  \begin{minipage}{0.7\textwidth}
    \begin{prooftree}
      \AxiomC{$\vdash \wellformed{P}$}
      \AxiomC{$\overline{x\prd\tau_i},\overline{\alpha\cnt\tau_j} \vdash,\mathbf{def}\ \text{f}(\overline{x_i\prd\tau_i};\overline{\alpha_j\cnt\tau_j})\coloneq s s$}
      \RightLabel{\textsc{Wf-Cons}}
      \BinaryInfC{$\vdash \wellformed{P,\mathbf{def}\ \text{f}(\overline{x_i\prd\tau_i},\overline{\alpha_j\cnt\tau_j}) \coloneq s}$}
    \end{prooftree}
  \end{minipage}



%%
%% Section: Translation
%%
\section{Translating Fun to Core}
\label{sec:translation}
In this section we show how to translate terms of the surface language \surfacelang{} to producers of the intermediate language \targetlang.
We first introduce a simpler translation $\translate{\cdot}$ in \cref{subsec:translation:naive} which is straightforward but introduces administrative redexes.
We then introduce the optimized translation $\translatestar{\cdot}{}$ in \cref{subsec:translation:optimized} whis is more complicated, but which does not generate 
any additional administrative redexes.
We finish in \cref{subsec:translation:properties} by proving some core properties of these translation functions.

\[
  \begin{array}{rcll}
    \multicolumn{4}{c}{\translate{\cdot{}}{} : \emph{Declaration}_{Fun} \rightarrow \emph{Declaration}_{Core}}\\
    \translate{\mathbf{data}\ T \{ \overline{K_i\ \Gamma_i} \}}{} & \coloneq & \mathbf{data} \{ \overline{K_i\ \translate{\Gamma_i}}\}\\
    \translate{\mathbf{codata}\ T \{ \overline{D_i\ \Gamma_i} : \tau \}}{} & \coloneq & \mathbf{codata} \{ \overline{D_i\ \translate{\Gamma_i},\alpha\cnt\tau}\} & \fresh{\alpha}\\
    \multicolumn{4}{c}{\translate{\cdot{}}{} : \emph{Typing Context}_{Fun} \rightarrow \emph{Typing Context}_{Core}}\\
    \translate{\nil}{} & \coloneq & \nil \\
    \translate{\Gamma,x:\tau}{} & \coloneq & \translate{\Gamma}, x\prd\tau\\
    \translate{\Gamma,\alpha\cnt\tau}{} & \coloneq & \translate{\Gamma}, \alpha\cnt\tau
  \end{array}
\]

Translating Declarations and typing contexts are the same for both translations we will introduce. 
Here, the most important part of the translation is the fact that codata declarations no longer have a type $\tau$ used for destructor terms.
Instead, each destructor gets a fresh covariable argument $\alpha$ with this type.
As we will see in the translations for cocases, this will ensure types are preserved under translation.

%%
%% Subsection: Naive Translation
%%
\subsection{Naive Translation}
\label{subsec:translation:naive}

First, we will introduce the naive translation, turning terms in \surfacelang{} into producers in \targetlang{}. 
As we can see from the definitions of \targetlang{} (\cref{sec:core}), certain terms, for example cases and destructors, are consumers in \targetlang{}, and others, such as if zero and top-level calls are statements.
In order to correctly translate such terms to \targetlang{}, we thus introduce $\mu$-abstractions, turning statements into producers.
Where \targetlang{}-expressions are consumers, they are cut with some other producer to turn them into statements which then make a producer with the $\mu$-abstraction.
We assume that each covariable $\alpha$ appearing in a translated term is fresh with respect to the term that is translated.
For example, when translating a top-level call $f(\overline{t_i};\overline{\alpha_i})$, the $\alpha_j$ used in the translation does not appear free in any of the $t_i$ and is not equal to any of the $\alpha_j$.
\[
  \begin{array}{rcl rcl}
    \multicolumn{6}{c}{\translate{\cdot} : \emph{Term} \rightarrow \emph{Producer}}\\
    \\
    \translate{\lit{x}} & \coloneq & \lit{x} & 
    \translate{\ifz{t_1}{t_2}{t_3}} & \coloneq & \mu\alpha.\ifz{\translate{t_1}}{\cut{\translate{t_2}}{\alpha}}{\cut{\translate{t_3}}{\alpha}} \\
    \translate{\lit{n}} & \coloneq & \lit{n}  &  
    \translate{t_1 \odot t_2} & \coloneq & \mu\alpha.\odot(\translate{t_1}, \translate{t_2}; \alpha)  \\
    \translate{f(\overline{t_i}; \overline{\alpha_j})} & \coloneq & \mu\alpha.f(\overline{\translate{t_i}}; \overline{\alpha_j},\alpha)  & 
    \translate{\letin{x}{t_1}{t_2}} & \coloneq & \mu\alpha.\cut{\translate{t_1}}{\tilde{\mu}x.\cut{\translate{t_2}}{\alpha}} \\
    \translate{K(\overline{t_i})} & \coloneq & K(\overline{\translate{t_i}}) &  
    \translate{\caseof{t}{\overline{K_i(\overline{x_{i,j}}) \Rightarrow t_i}}} & \coloneq & \mu\alpha.\cut{\translate{t}}{\case{\overline{K_i(\overline{x_{i,j}}) \Rightarrow \cut{\translate{t_i}}{\alpha}}}}  \\
    \translate{t.D(\overline{t_i})} & \coloneq & \mu\alpha.\cut{\translate{t}}{D(\overline{\translate{t_i}}; \alpha)}   & 
    \translate{\cocase{\overline{D_i(\overline{x_{ij}})\Rightarrow t_i}}} & \coloneq & \cocase{\overline{D_i(\overline{x_{i,j}}; \alpha_i) \Rightarrow \cut{\translate{t_i}}{\alpha_i}}}  \\
    \translate{\lab{\alpha}{t}} & \coloneq & \mu\alpha.\cut{\translate{t}}{\alpha} & 
    \translate{\goto{t}{\alpha}} & \coloneq & \mu\beta.\cut{\translate{t}}{\alpha}  \\
    \\
    \multicolumn{6}{c}{\translate{\cdot{}}{} : \emph{Definition}_{Fun} \rightarrow \emph{Definition}_{Core}}\\
    \multicolumn{2}{r}{\translate{\defi{f(\overline{x}; \overline{\alpha})}{t}}{}} & \multicolumn{2}{c}{\coloneq} & \multicolumn{2}{l}{\defi{f(\overline{x}; \overline{\alpha}, \alpha)}{\cut{\translate{t}}{\alpha}}} \\
    \\
  \end{array}
\]
In order to see why this translation is inefficient, consider the following term and its translation.
\begin{example}
  \label{ex:translation_naive}
  \begin{align*}
    \translate{\letin{x}{\lit{2}}{x*x}} = \mu\alpha.\cut{\lit{2}}{\tilde\mu x.\cut{\mu\beta.*(x,x;\beta)}{\alpha}}
  \end{align*}
  Here, the statement bound by $\tilde\mu$ is the cut $\cut{\mu\beta.*(x,x;\beta)}{\alpha}$, which after a single reduction step becomes $*(x,x;\alpha)$.
\end{example}
Thus, whenever we introduce a new $\mu$-abstraction while translating a term in \surfacelang{}, there is a potential administrative redex generated  depending on the surrounding terms.
With these redexes included, we have two options on how to continue compilation. 
Either we keep them in the expressions and finally translate them into machine code, or we add an additional simplification step before we translate \targetlang{} to AxCut.
In each case, we will incur a performance overhead which we would like to avoid. 
To solve this issue, we will introduce the optimized translation, which keeps track of introduced covariables and only generate new ones where necessary.


%%
%% Subsection: Optimized Translation
%%
\subsection{Optimized Translation}
\label{subsec:translation:optimized}
We now introduce the optimized version of the translation introduced above.
This optimization works by splitting the function $\translate{\cdot}$ into the two functions $\translatestar{\cdot}{}$ and $\translatestar{\cdot}{k}$.
The intuition is that $\translatestar{p}{k}$ should be the same as $\cut{\translate{p}}{k}$, but whenever this results in a redex, for example if $\translate{p}$ has the form $\mu \alpha.s$, we reduce this administrative redex during the translation itself.
As before, we will again assume that any covariable $\alpha$ in the translation of a term is fresh with respect to the term being translated.

\[
  \begin{array}{rcl rcl}
    \multicolumn{6}{c}{\translatestar{\cdot}{} : \emph{Term} \rightarrow  \emph{Producer}}\\
    \translatestar{x}{} & \coloneqq & x & 
    \translatestar{\cocase{\overline{D_i(\overline{x_{i,j}}) \Rightarrow t_i}}}{} & \coloneq & \cocase{\overline{D_i(\overline{x_{i,j}}; \alpha_i) \Rightarrow \translatestar{t_i}{\alpha_i}}} \\
    \translatestar{\lit{n}}{} & \coloneqq & \lit{n}  &
    \translatestar{\lambda x.t}{} & \coloneq & \cocase{\mathtt{ap}(x; \alpha) \Rightarrow \translatestar{t}{\alpha}} \\
    \translatestar{K(\overline{t_i})}{} & \coloneqq & K(\overline{\translatestar{t_i}{}}) &
    \translatestar{\lab{\alpha}{t}}{} & \coloneq & \mu \alpha.\translatestar{t}{\alpha}{} \\
    \translatestar{t}{} & \coloneqq & \mu\alpha.\translatestar{t}{\alpha} & \fresh{\alpha} &
    \\
    \multicolumn{6}{c}{\translatestar{\cdot}{\cdot} : \emph{Term} \times \emph{Consumer} \rightarrow \emph{Statement}}\\
    \translatestar{x}{c} & \coloneq & \cut{x}{c} & 
    \translatestar{t_1 \odot t_2}{c} & \coloneq & \odot(\translatestar{t_1}{}, \translatestar{t_2}{}; c)  \\
    \translatestar{\lit{n}}{c} & \coloneq & \cut{\lit{n}}{c} & 
    \translatestar{\ifz{t_1}{t_2}{t_2}}{c} & \coloneq & \ifz{\translatestar{t_1}{}}{\translatestar{t_2}{c}}{\translatestar{t_3}{c}}  \\
    \translatestar{\letin{x}{t_1}{t_2}}{c} & \coloneq & \translatestar{t_1}{\tilde{\mu}x.\translatestar{t_2}{c}} & 
    \translatestar{f(\overline{t_i}; \overline{\alpha_j})}{c} & \coloneq & f(\overline{\translatestar{t_i}{}}; \overline{\alpha_j}, c) \\
    \translatestar{K(\overline{t_i})}{c} & \coloneq & \cut{K(\overline{\translatestar{t_i}{}})}{c} & 
    \translatestar{\caseof{t}{\overline{K_i(\overline{x_{i,j}}) \Rightarrow t_i}}}{c} & \coloneq & \translatestar{t}{\case{\overline{K_i(\overline{x_{i,j}}) \Rightarrow \translatestar{t_i}{c}}}} \\
    \translatestar{t.D(\overline{t_i})}{c} & \coloneq & \translatestar{t}{D(\overline{\translatestar{t_i}{}}; c)} &
    \translatestar{\cocase{\overline{D_i(\overline{x_{i,j}}) \Rightarrow t_i}}}{c} & \coloneq & \cut{\cocase{\overline{D_i(\overline{x_{i,j}}; \alpha_i) \Rightarrow \translatestar{t_i}{\alpha_i}}}}{c}  \\
    \translatestar{\lab{\alpha}{t}}{c} & \coloneq & \cut{\mu \alpha.\translatestar{t}{\alpha}}{c} & 
    \translatestar{\goto{t}{\alpha}}{c} & \coloneq & \translatestar{t}{\alpha} \\
    \\
    \multicolumn{6}{c}{\translatestar{\cdot}{} : \emph{Definition} \rightarrow \emph{Definition}}\\
    \multicolumn{2}{r}{\translatestar{\defi{f(\overline{x}; \overline{\alpha})}{t}}{}} & \multicolumn{2}{c}{\coloneq} & \multicolumn{2}{l}{\defi{f(\overline{x}; \overline{\alpha}, \alpha)}{\translatestar{t}{\alpha}}} 
  \end{array}
\]
To see how this translation removes administrative redexes, consider again the example \cref{example:translation:naive}
\begin{example}
  \begin{align*}
    \translate{\letin{x}{\lit{2}}{x*x}}^* = \mu\alpha.\cut{\lit{2}}{\tilde\mu x.*(x,x;\alpha)}
  \end{align*}
  The resulting producer in \targetlang{} is exactly the same as the one generated from the naive translation after reducing the administrative redex.
\end{example}
As we will see below (\cref{teo:correctness}), this property holds in general.

%%
%% Section Properties
%%
\subsection{Properties}
\label{subsec:translation:properties}

\begin{lemma}
  If $\Gamma \vdash t : \tau$ and $\Gamma' \vdash k \cnt \tau$, then $\Gamma' \vdash \translatestar{t}{k}$.
\end{lemma}
\begin{proof}
  TODO: Figure out correct statement w.r.t. $\Gamma$ and $\Gamma'$.
\end{proof}

\begin{theorem}[Type Preservation]
  Let $t$ be a term in \surfacelang, and let $\tau$ be a type.
  If $\Gamma \vdash t: \tau$, then both $\Gamma \vdash \translate{t} \prd \tau$ and $\Gamma \vdash \translatestar{t}{} \prd \tau$.
\end{theorem}

\begin{theorem}[Correctness]
  \label{teo:correctness}
  Let $t$ be a term in \surfacelang. Then there are a finite number of reduction steps (where reduction here also means reduction under binders) such that $\focus{\translate{t}} \reducesto^{\ast} \focus{\translatestar{t}{}}$, where $\focus{\cdot}$ is the focusing translation.
\end{theorem}


%%
%% Section: Naming Transformation
%%
\section{Naming Transformation}
\label{sec:naming-transformation}
This is a generalization of the focusing transformation which names all subterms instead of only non-value producers.
It targets the following fragment \targetvar{} of \targetlang{} where arguments are always (co-)variables

\begin{definition}[Syntax of \targetvar{}]
  \[
    \begin{array}{rcll}
      p & \coloneqq & x \mid \lit{n} \mid \mu\alpha.s \mid K\ \Gamma \mid \cocase{\overline{D\ \Gamma \Rightarrow s}} & \emph{Producers}\\
      c & \coloneqq & \alpha \mid \tilde{\mu}x.s \mid D\ \Gamma \mid \case{\overline{K\ \Gamma \Rightarrow s}}& \emph{Consumers}\\
      s & \coloneqq & \cut{p}{c} \mid \ifz{x}{s}{s} \mid \odot(x, x; \alpha) \mid f\ \Gamma \mid \done & \emph{Statements}\\
    \end{array}
  \]
\end{definition}


The cases for constructors and destructors are special-cased to avoid administrative redexes.

\[
  \begin{array}{rcll}
    \\
    \multicolumn{4}{c}{\name{\cdot} : \emph{Definition}_{\targetlang{}} \rightarrow \emph{Definition}_{\targetvar{}}}\\
    \name{\defi{f\ \Gamma}{s}} & \coloneq & \defi{f\ \Gamma}{\name{s}} & \\
    \\
    \multicolumn{4}{c}{\name{\cdot} : \emph{Statement}_{\targetlang{}} \rightarrow \emph{Statement}_{\targetvar{}}}\\
    \name{\cut{K\ \sigma}{c}} & \coloneq & \binds{\sigma}{\lambda as.\cut{K(as)}{\name{c}}} & \\
    \name{\cut{p}{D\ \sigma}} & \coloneq & \binds{\sigma}{\lambda as.\cut{\name{p}}{D(as)}} & \\
    \name{\cut{p}{c}} & \coloneq & \cut{\name{p}}{\name{c}} & \\
    \name{\odot(p_1, p_2; c)}{} & \coloneq & \bind{p_1}{\lambda a_1.\bind{p_2}{\lambda a_2.\odot(a_1, a_2; c)}} & \\
    \name{\ifz{p}{s_1}{s_2}}{} & \coloneq & \bind{p}{\lambda a.\ifz{a}{\name{s_1}}{\name{s_2}}} & \\
    \name{f\ \sigma}{} & \coloneq & \binds{\sigma}{\lambda as.f(as)} & \\
    \name{\done}{} & \coloneq & \done & \\
    \\
    \multicolumn{4}{c}{\name{\cdot} : \emph{Producer}_{\targetlang{}} \rightarrow \emph{Producer}_{\targetvar{}}}\\
    \name{x} & \coloneq & x & \\
    \name{\lit{n}} & \coloneq & \lit{n} & \\
    \name{\mu\alpha.s} & \coloneq & \mu\alpha.\name{s} & \\
    \name{K\ \sigma} & \coloneq & \mu\alpha.\binds{\sigma}{\lambda as.\cut{K(as)}{\alpha}} & \fresh{\alpha} \\
    \name{\cocase{\overline{D_i\ \Gamma_i \Rightarrow s_i}}} & \coloneq & \cocase{\overline{D_i\ \Gamma_i \Rightarrow \name{s_i}}} & \\
    \\
    \multicolumn{4}{c}{\name{\cdot} : \emph{Consumer}_{\targetlang{}} \rightarrow \emph{Consumer}_{\targetvar{}}}\\
    \name{\alpha} & \coloneq & \alpha & \\
    \name{\tilde{\mu}x.s} & \coloneq & \tilde{\mu}x.\name{s} & \\
    \name{D\ \sigma} & \coloneq & \tilde{\mu}x.\binds{\sigma}{\lambda as.\cut{x}{D(as)}} & \fresh{x} \\
    \name{\case{\overline{K_i\ \Gamma_i \Rightarrow s_i}}} & \coloneq & \case{\overline{K_i\ \Gamma_i \Rightarrow \name{s_i}}} & \\
  \end{array}
\]

\[
  \begin{array}{rcll}
    \multicolumn{4}{c}{\bind{\cdot}{\cdot} : \emph{Producer}_{\targetlang{}} \times (\emph{Name} \rightarrow \emph{Statement}_{\targetvar{}}) \rightarrow \emph{Statement}_{\targetvar{}}}\\
    \bind{x}{k} & \coloneq & k(x) & \\
    \bind{\lit{n}}{k} & \coloneq & \cut{\lit{n}}{\tilde{\mu}x.k(x)} & \fresh{x} \\
    \bind{\mu\alpha.s}{k} & \coloneq & \cut{\mu\alpha.\name{s}}{\tilde{\mu}x.k(x)} & \fresh{x} \\
    \bind{K\ \sigma}{k} & \coloneq & \binds{\sigma}{\lambda as.\cut{K(as)}{\tilde{\mu}x.k(x)}} & \fresh{x} \\
    \bind{\cocase{\overline{D_i\ \Gamma_i \Rightarrow s_i}}}{k} & \coloneq & \cut{\cocase{\overline{D_i\ \Gamma_i \Rightarrow \name{s_i}}}}{\tilde{\mu}x.k(x)} & \fresh{x} \\
    \\
    \multicolumn{4}{c}{\bind{\cdot}{\cdot} : \emph{Consumer}_{\targetlang{}} \times (\emph{Name} \rightarrow \emph{Statement}_{\targetvar{}}) \rightarrow \emph{Statement}_{\targetvar{}}}\\
    \bind{\alpha}{k} & \coloneq & k(\alpha) & \\
    \bind{\tilde{\mu}x.s}{k} & \coloneq & \cut{\mu\alpha.k(\alpha)}{\tilde{\mu}x.\name{s}} & \fresh{\alpha} \\
    \bind{D\ \sigma}{k} & \coloneq & \binds{\sigma}{\lambda as.\cut{\mu\alpha.k(\alpha)}{D(as)}} & \fresh{\alpha} \\
    \bind{\case{\overline{K_i\ \Gamma_i \Rightarrow s_i}}}{k} & \coloneq & \cut{\mu\alpha.k(\alpha)}{\case{\overline{K_i\ \Gamma_i \Rightarrow \name{s_i}}}} & \fresh{\alpha} \\
    \\
    \multicolumn{4}{c}{\binds{\cdot}{\cdot} : \emph{Substitution} \times (\emph{List Name} \rightarrow \emph{Statement}_{\targetvar{}}) \rightarrow \emph{Statement}_{\targetvar{}}}\\
    \binds{\nil}{k} & \coloneq & k(\nil) & \\
    \binds{e :: es}{k} & \coloneq & \bind{e}{\lambda a.\binds{es}{\lambda as.k(a :: as)}} & \\
  \end{array}
\]


%%
%% Section: Redundancy Elimination
%%
\section{Redundancy Elimination}
\label{sec:redundancy-elimination}
The next step towards our normal consists of removing redundant part of the language.
More precisely, note that the fragment \targetvar{} explained in the previous section could as well be described with only statements, by inlining the producers and consumers into the cut statement and the consumers into the arithmetic operators.
This is because all producer and consumer arguments are (co-)variables in this fragment.
For arithmetic operators, typing only allows the consumer to be a covariable or a $\tilde\mu$-binding.
For cuts, we could in principle have 20 different forms of cuts since we have five different kinds of producers and four different kinds of consumers.
However, typing again precludes four of them: a constructor and a destructor as well as a pattern match and a copattern match can never meet in a cut, and moreover, an integer literal can never be cut with a destructor or a pattern match.
Together with conditionals, calls of top-level definitions and the terminating statement, this results in 21 statement forms in total.

We will now transform away eight of them, resulting in the following fragment \targetred{} of \targetlang{}.

\begin{definition}[Syntax of \targetred{}]
  \[
    \begin{array}{rcll}
      s & \coloneqq & \cut{K\ \Gamma}{\tilde{\mu}x.s} \mid \cut{x}{\case{\overline{K\ \Gamma \Rightarrow s}}} \mid \cut{\cocase{\overline{D\ \Gamma \Rightarrow s}}}{\tilde{\mu}x.s} \mid \cut{x}{D\ \Gamma} & \emph{Statements}\\
       & \mid & \cut{\mu\alpha.s}{D\ \Gamma} \mid \cut{\cocase{\overline{D\ \Gamma \Rightarrow s}}}{\alpha} \mid \cut{\mu\alpha.s}{\case{\overline{K\ \Gamma \Rightarrow s}}} \mid \cut{K\ \Gamma}{\alpha} & \\
       & \mid & \cut{\lit{n}}{\tilde{\mu}x.s} \mid \ifz{x}{s}{s} \mid \odot(x, x; \alpha) \mid f\ \Gamma \mid \done & \\
    \end{array}
  \]
\end{definition}

We consider the interesting cases in turn.
Let us start with the cases where a variable or covariable meets a $\tilde\mu$- or $\mu$-binding in a cut.
This is a simple renaming which we can transform away by reducing the cut via substitution.
Next, consider the cases of completely known cuts, i.e., cuts where one side is a constructor or destructor and the other side is a pattern match or copattern match.
These cuts can also simply be reduced.
As the arguments of constructors and destructors are all (co-)variables, this just amounts to picking the corresponding branch and renaming the bound (co-)variables.
Hence, there is no risk of non-termination.
Now, consider the case where a $\mu$- and a $\tilde\mu$-binding are cut, the famous critial pairs \cite{Curien2000duality}.
To transform those away, we distinguish between the different kinds of types.
For data and codata types, we have ensured that all $\eta$-laws hold by using call-by-value evaluation for data types and call-by-name evaluation for codata types.
Therefore, we can $\eta$-expand the $\tilde\mu$-binding in critical pairs of data types and the $\mu$-binding in critical pairs of data types.
For the primitive type $\tyint$, we cannot use $\eta$-expansion, so we need another way to resolve critical pairs.
The $\tilde\mu$-binding can be viewed as a continuation for integers.
As we use call-by-value evaluation for integers, we should find a different possibility to model this continuation, in such a way that the $\mu$-binding would be reduced first.
To do so, we can define a new data type:

\[
  \data{\mathtt{Cont}}{\mathtt{Ret(x : \tyint)}}
\]

\noindent A pattern match of this data type can also be viewed as a continuation for integers, the only difference from the $\tilde\mu$-binding being that the integer is wrapped into the constructor $\mathtt{Ret}$.
This means that in all places where an integer is cut with a covariable, which now stands for a consumer of type $\mathtt{Cont}$ instead of a $\tilde\mu$-binding of type $\tyint$, we have to wrap the integer into a $\mathtt{Ret}$.
There are three cases where this happens: in a cut of a variable and a covariable of type $\tyint$, in the cut of an integer literal with a covariable and in the case where the consumer of an arithmetic operator is a covariable.
For the latter two cases, we further insert a $\tilde\mu$-binding to give the integer literal or the result of the arithmetic operator a name before wrapping this name into a $\mathtt{Ret}$ which is then cut with the covariable.
It will become clear later, that no integer is ever actually wrapped into a constructor in the sense of storing it on the heap.
This is because those wrapped integers only appear in cuts with covariables in which the role of the constructor is to pick a branch in the pattern match.
As there is always only one branch, there is no overhead in modelling continuations for integers this way.
Finally, we transform away completely unknown cuts, that is, cuts of a variable with a covariable.
We have already seen above how to do this for primitive integers.
For data types and codata types we can again use $\eta$-expansion as in the cases of critical pairs, i.e., we $\eta$-expand the covariable in cuts at data types and we $\eta$-expand the variable in cuts at codata types.
All other cases are simple congruences.

\begin{gather*}
  \begin{array}{rcl}
    \multicolumn{3}{c}{\redundancy{\cdot} : \emph{Definition}_{\targetvar{}} \rightarrow \emph{Definition}_{\targetred{}}}\\
    \redundancy{\defi{f\ \Gamma}{s}} & \coloneq & \defi{f\ \Gamma}{\redundancy{s}} \\
  \end{array}
  \\\\
  \begin{array}{rclrcl}
    \multicolumn{6}{c}{\redundancy{\cdot} : \emph{Statement}_{\targetvar{}} \rightarrow \emph{Statement}_{\targetred{}}}\\
    \redundancy{\cut{\mu\alpha.s}{\beta}} & \coloneq & \redundancy{s[\alpha \mapsto \beta]} &
    \redundancy{\cut{K_j\ \Gamma_0}{\case{\overline{K_i\ \Gamma_i \Rightarrow s_i}}}} & \coloneq & \redundancy{s_j[\Gamma_j \mapsto \Gamma_0]} \\
    \redundancy{\cut{y}{\tilde{\mu}x.s}} & \coloneq & \redundancy{s[x \mapsto y]} & \qquad
    \redundancy{\cut{\cocase{\overline{D_i\ \Gamma_i \Rightarrow s_i}}}{D_j\ \Gamma_0}} & \coloneq & \redundancy{s_j[\Gamma_j \mapsto \Gamma_0]} \\
  \end{array}
  \\
  \begin{array}{rcl}
    \redundancy{\cut{\mu\alpha.s_1}{\tilde{\mu}x.s_2}_T} & \coloneq & \cut{\mu\alpha.\redundancy{s_1}}{\case{\overline{K_i\ \Gamma_i \Rightarrow \cut{K_i\ \Gamma_i}{\tilde{\mu}x.\redundancy{s_2}}}}} \\
    \text{where} &  & \data{T}{\overline{K_i\ \Gamma_i}} \in \Theta \quad \fresh{\overline{\Gamma_i}} \\
    \redundancy{\cut{\mu\alpha.s_1}{\tilde{\mu}x.s_2}_T} & \coloneq & \cut{\cocase{\overline{D_i\ \Gamma_i \Rightarrow \cut{\mu\alpha.\redundancy{s_1}}{D_i\ \Gamma_i}}}}{\tilde{\mu}x.s_2} \\
    \text{where} &  & \codata{T}{\overline{D_i\ \Gamma_i}} \in \Theta \quad \fresh{\overline{\Gamma_i}} \\
    \redundancy{\cut{\mu\alpha.s_1}{\tilde{\mu}x.s_2}_\tyint} & \coloneq & \cut{\mu\alpha.s_1}{\case{\mathtt{Ret}(x) \Rightarrow \redundancy{s_2}}} \\
    \text{where} & & \data{\mathtt{Cont}}{\mathtt{Ret(x : \tyint)}} \in \Theta \\
    \redundancy{\cut{x}{\alpha}_T} & \coloneq & \cut{x}{\case{\overline{K_i\ \Gamma_i \Rightarrow \cut{K_i\ \Gamma_i}{\alpha}}}} \\
    \text{where} &  & \data{T}{\overline{K_i\ \Gamma_i}} \in \Theta \quad \fresh{\overline{\Gamma_i}} \\
    \redundancy{\cut{x}{\alpha}_T} & \coloneq & \cut{\cocase{\overline{D_i\ \Gamma_i \Rightarrow \cut{x}{D_i\ \Gamma_i}}}}{\alpha} \\
    \text{where} &  & \codata{T}{\overline{D_i\ \Gamma_i}} \in \Theta \quad \fresh{\overline{\Gamma_i}} \\
    \redundancy{\cut{x}{\alpha}_\tyint} & \coloneq & \cut{\mathtt{Ret}(x)}{\alpha} \\
  \end{array}
  \\
  \begin{array}{rclrcl}
    \redundancy{\cut{\lit{n}}{\alpha}} & \coloneq & \cut{\lit{n}}{\tilde{\mu}x.\cut{\mathtt{Ret}(x)}{\alpha}} &
    \redundancy{\odot(x_1, x_2; \alpha)}{} & \coloneq & \odot(x_1, x_2; \tilde{\mu}x.\cut{\mathtt{Ret}(x)}{\alpha}) \\
    \text{where} &  & \fresh{x} &
    \text{where} &  & \fresh{x} \\
    \redundancy{\cut{\lit{n}}{\tilde{\mu}x.s}} & \coloneq & \cut{\lit{n}}{\tilde{\mu}x.\redundancy{s}} &
    \redundancy{\odot(x_1, x_2; \tilde{\mu}x.s)}{} & \coloneq & \odot(x_1, x_2; \tilde{\mu}x.\redundancy{s}) \\
    \redundancy{\cut{K\ \Gamma_0}{\tilde{\mu}x.s}} & \coloneq & \cut{K\ \Gamma_0}{\tilde{\mu}x.\redundancy{s}} &
    \redundancy{\cut{x}{\case{\overline{K_i\ \Gamma_i \Rightarrow s_i}}}} & \coloneq & \cut{x}{\case{\overline{K_i\ \Gamma_i \Rightarrow \redundancy{s_i}}}} \\
    \redundancy{\cut{\mu\alpha.s}{D\ \Gamma_0}} & \coloneq & \cut{\mu\alpha.\redundancy{s}}{D\ \Gamma_0} &
    \redundancy{\cut{\cocase{\overline{D_i\ \Gamma_i \Rightarrow s_i}}}{\alpha}} & \coloneq & \cut{\cocase{\overline{D_i\ \Gamma_i \Rightarrow \redundancy{s_i}}}}{\alpha} \\
    \redundancy{\cut{x}{D\ \Gamma_0}} & \coloneq & \cut{x}{D\ \Gamma_0} &
    \redundancy{\cut{\cocase{\overline{D_i\ \Gamma_i \Rightarrow s_i}}}{\tilde{\mu}x.s}} & \coloneq & \cut{\cocase{\overline{D_i\ \Gamma_i \Rightarrow \redundancy{s_i}}}}{\tilde{\mu}x.\redundancy{s}} \\
    \redundancy{\cut{K\ \Gamma_0}{\alpha}} & \coloneq & \cut{K\ \Gamma_0}{\alpha} &
    \redundancy{\cut{\mu\alpha.s}{\case{\overline{K_i\ \Gamma_i \Rightarrow s_i}}}} & \coloneq & \cut{\mu\alpha.\redundancy{s}}{\case{\overline{K_i\ \Gamma_i \Rightarrow \redundancy{s_i}}}} \\
    \redundancy{f\ \Gamma_0}{} & \coloneq & f\ \Gamma_0 &
    \redundancy{\ifz{x}{s_1}{s_2}}{} & \coloneq & \ifz{x}{\redundancy{s_1}}{\redundancy{s_2}} \\
    \redundancy{\done}{} & \coloneq & \done & & & \\
  \end{array}
\end{gather*}


%%
%% Section: AxCut
%%
\section{The Language AxCut}
\label{sec:axcut}

\begin{definition}[Syntax of AxCut]
    \[ 
      \begin{array}{r c l l}
        s & \coloneqq & \done  \mid \jump{f} \mid \substitute{v \mapsto v,\ldots}{s} \mid \letac{v}{m(\Gamma)}{s} \mid \switch{v}{b} \mid \new{v}{(\Gamma)b}{s} \mid \invoke{v}{m} & \emph{Statements}\\
        \Theta & \coloneqq & \emptyset \mid \defi{f(\Gamma)}{s}; \Theta & \emph{Programs}\\
      \end{array}
    \]
\end{definition}

%%
%% Subsec: Typing Rules
%%
\subsection{Typing Rules}
\label{subsec:axcut:typing-rules}



%%
%% Section: Translation to AxCut
%%
\section{Translation to AxCut}
\label{sec:toaxcut}
This is the translation from \targetsub{} to \machinelang.
It collapses data and codata types and renames constructs to a unified syntax.

\[
  \begin{array}{rcll}
    \multicolumn{4}{c}{\collapse{\cdot} : \emph{Type}_{\targetsub{}} \rightarrow \emph{Type}_{\machinelang{}}}\\
    \collapse{\data{T}{\overline{K_i\ \Gamma_i}}}{} & \coloneq & \sig{T}{\overline{K_i\ \collapse{\Gamma_i}}} & \\
    \collapse{\codata{T}{\overline{D_i\ \Gamma_i}}}{} & \coloneq & \sig{T}{\overline{D_i\ \collapse{\Gamma_i}}} & \\
    \collapse{\Gamma, v : T}{} & \coloneq & \collapse{\Gamma}{}, v\ \collapse{: T} & \\
    \collapse{: T} & \coloneq & : T & \text{where } T : \mathbf{data} \\
    \collapse{\cnt T} & \coloneq & \cnt T & \text{where } T : \mathbf{data} \\
    \collapse{: T} & \coloneq & \cnt T & \text{where } T : \mathbf{codata} \\
    \collapse{\cnt T} & \coloneq & : T & \text{where } T : \mathbf{codata} \\
    \collapse{: \tyint} & \coloneq & \ext \tyint & \\
    \\
    \multicolumn{4}{c}{\collapse{\cdot} : \emph{Definition}_{\targetsub{}} \rightarrow \emph{Definition}_{\machinelang{}}}\\
    \collapse{\defi{f\ \Gamma}{s}} & \coloneq & \defi{f : \Gamma}{\collapse{s}} & \\
    \\
    \multicolumn{4}{c}{\collapse{\cdot} : \emph{Statement}_{\targetsub{}} \rightarrow \emph{Statement}_{\machinelang{}}}\\
    \collapse{\cut{K\ \Gamma_0}{\tilde{\mu}x.s}} & \coloneq & \letac{x}{K\ \Gamma_0}{\collapse{s}} & \\
    \collapse{\cut{\mu\alpha.s}{D\ \Gamma_0}} & \coloneq & \letac{\alpha}{D\ \Gamma_0}{\collapse{s}} & \\
    \collapse{\cut{K\ \Gamma_0}{\alpha}} & \coloneq & \invoke{\alpha}{K} & \\
    \collapse{\cut{x}{D\ \Gamma_0}} & \coloneq & \invoke{x}{D} & \\
    \collapse{\cut{x}{\case{\overline{K_i\ \Gamma_i \Rightarrow s_i}}}} & \coloneq & \switch{x}{\bra{\overline{K_i\ \Gamma_i \Rightarrow \collapse{s_i}}}} & \\
    \collapse{\cut{\cocase{\overline{D_i\ \Gamma_i \Rightarrow s_i}}}{\alpha}} & \coloneq & \switch{\alpha}{\bra{\overline{D_i\ \Gamma_i \Rightarrow \collapse{s_i}}}} & \\
    \collapse{\cut{\mu\alpha.s}{\case{\overline{K_i\ \Gamma_i \Rightarrow s_i}}_{\Gamma_0}}} & \coloneq & \new{\alpha}{(\Gamma_0)\bra{\overline{K_i\ \Gamma_i \Rightarrow \collapse{s_i}}}}{\collapse{s}} & \\
    \collapse{\cut{\cocase{\overline{D_i\ \Gamma_i \Rightarrow s_i}}_{\Gamma_0}}{\tilde{\mu}x.s}} & \coloneq & \new{x}{(\Gamma_0)\bra{\overline{D_i\ \Gamma_i \Rightarrow \collapse{s_i}}}}{\collapse{s}} & \\
    \collapse{\cut{\lit{n}}{\tilde{\mu}x.s}} & \coloneq & \litac{n}{x}{\collapse{s}} & \\
    \collapse{\odot(x_1, x_2; \tilde{\mu}x.s)} & \coloneq & \opac{x_1, x_2}{x}{\collapse{s}} & \\
    \collapse{\ifz{x}{s_1}{s_2}} & \coloneq & \ifzac{x}{\collapse{s_1}}{\collapse{s_2}} & \\
    \collapse{\substitute{\Gamma \mapsto \Gamma^{\prime}}{s}} & \coloneq & \substituteac{\Gamma \mapsto \Gamma^{\prime}}{\collapse{s}} & \\
    \collapse{f\ \Gamma_0} & \coloneq & \jump{f} & \\
    \collapse{\done} & \coloneq & \done & \\
  \end{array}
\]


%%
%% Section: Related Work
%%
\section{Related Work}
\label{sec:related-work}
\begin{itemize}
    \item \cite{Downen2016sequent}
\end{itemize}


%%
%% Section: Conclusion
%%
\section{Conclusion}
\label{sec:conclusion}
\input{sections/conclusion.tex}

%%
%% Bibliography
%%
\bibliography{bibliography/bibliography.bib, bibliography/ownpublications.bib}

\end{document}
