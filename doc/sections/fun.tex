\begin{definition}[Syntactic Conventions]
  We use the following metavariables for all languages:
  \[
    \begin{array}{rcll}
      \odot  & \coloneqq & + \mid - \mid * & \emph{Arithmetic Operators}
    \end{array}
  \]
\end{definition}

\begin{definition}[Syntax of \surfacelang{}]
  We assume an infinite set of names $\mathcal{N}$ containing type names $T\in\mathcal{N}$, constructor names $K\in\mathcal{N}$ and destructor names $D\in\mathcal{N}$.
  \[
    \begin{array}{r c l l}
      t & \coloneqq & x \mid \lit{n} \mid t \odot t \mid \ifz{t}{t}{t} \mid \letin{x}{t}{t} \mid \lab{\alpha}{t} \mid \goto{t}{\alpha}  & \emph{Terms}\\
      & \mid & K(\sigma) \mid \caseof{t}{\overline{K\ \Gamma \Rightarrow t}} \mid t.D(\sigma) \mid \cocase{\overline{D\ \Gamma \Rightarrow t}} \mid f(\sigma)& \\
      c & \coloneqq & \alpha & \emph{Consumers}\\
      \sigma & \coloneqq & \nil \mid \sigma,t \mid \sigma, c & \emph{Substitutions}\\
      \tau & \coloneqq & \tyint \mid T & \emph{Types} \\
      \kappa & \coloneq & \mathbf{prim}\mid\mathbf{data} \mid \mathbf{codata} & \emph{Kinds} \\
      \Gamma & \Coloneqq & \nil \mid \Gamma, x : \tau \mid \Gamma, \alpha \cnt \tau & \emph{Typing Contexts} \\
      \delta & \coloneqq & \mathbf{data}\ T\ \{ \overline{K\ \Gamma} \}  \mid \mathbf{codata}\ T\ \{ \overline{D\ \Gamma : \tau}\} \mid \defi{f\ \Gamma}{t} & \emph{Declarations}\\
      \Theta & \coloneqq & \overline{\delta} & \emph{Programs}\\
    \end{array}
  \]
\end{definition}
In both the definitions for (co-) cases and for declarations we use typing contexts $\Gamma$ to encode the arguments of constructors and destructors.
This is a slight abuse of notation, since in data declarations, types for arguments are mandatory, while for clauses in (co-) cases, they are omitted.
Thus, the following is a valid program in \surfacelang{}.
\begin{example}
  \label{ex:fun-syntax}
  \begin{lstlisting}
  data ListInt {
  Nil,
  Cons(x:Int,xs:ListInt)
  }

  def tl(ls) := case ls of {
  Nil => Nil,
  Cons(x,xs) => xs
  }
  \end{lstlisting}
\end{example}
In the above example, the constructor $\mathtt{Cons}$ has arguments $x$ and $xs$, both of which are type annotated in the declaration of $\mathtt{ListInt}$.

%%
%% Subsec: Typing Rules
%%
\subsection{Typing Rules}
\label{subsec:fun:typing-rules}

\begin{figure}[hbtp]
  \begin{minipage}{\textwidth}
  \vspace{1em}
  \judgementbox{Typing Terms}{$\Theta\mid\Gamma\vdash t : \tau$}

  \begin{minipage}{0.25\textwidth}
    \begin{prooftree}
      \AxiomC{$x : \tau \in \Gamma$}
      \RightLabel{\textsc{Var}}
      \UnaryInfC{$\Theta\mid\Gamma \vdash x : \tau$}
    \end{prooftree}
  \end{minipage}
  \hfill
  \begin{minipage}{0.25\textwidth}
    \begin{prooftree}
      \AxiomC{\phantom{$x : \tau \in \Gamma$}}
      \RightLabel{\textsc{Lit}}
      \UnaryInfC{$\Theta\mid\Gamma \vdash \lit{n}:\tyint$}
    \end{prooftree}
  \end{minipage}
  \hfill
  \begin{minipage}{0.4\textwidth}
    \begin{prooftree}
      \AxiomC{$\Theta\mid\Gamma \vdash t_1: \tyint \quad \Theta\mid\Gamma \vdash t_2: \tyint$}
      \RightLabel{\textsc{Op}}
      \UnaryInfC{$\Theta\mid\Gamma \vdash t_1\odot t_2 : \tyint$}
    \end{prooftree}
  \end{minipage}
  \hfill
  \vspace{1em}

  \begin{minipage}{0.45\textwidth}
    \begin{prooftree}
      \AxiomC{$\Theta\mid\Gamma \vdash t : \tyint$}
      \AxiomC{$\Theta\mid\Gamma \vdash t_1 : \tau$}
      \AxiomC{$\Theta\mid\Gamma \vdash t_2 : \tau$}
      \RightLabel{\textsc{Ifz}}
      \TrinaryInfC{$\Theta\mid\Gamma \vdash \ifz{t}{t_1}{t_2} : \tau$}
    \end{prooftree}
  \end{minipage}
  \hfill
  \begin{minipage}{0.45\textwidth}
    \begin{prooftree}
      \AxiomC{$\Theta\mid\Gamma \vdash t_1 : \tau_1$}
      \AxiomC{$\Theta\mid\Gamma, x : \tau_1 \vdash t_2 : \tau_2$}
      \RightLabel{\textsc{Let}}
      \BinaryInfC{$\Theta\mid\Gamma \vdash \letin{x}{t_1}{t_2} :\tau_2$}
    \end{prooftree}
  \end{minipage}
  \hfill
  \vspace{1em}

  \begin{minipage}{0.4\textwidth}
    \begin{prooftree}
      \AxiomC{$\Theta\mid\Gamma, \alpha \cnt \tau \vdash t : \tau$}
      \RightLabel{\textsc{Label}}
      \UnaryInfC{$\Theta\mid\Gamma \vdash \lab{\alpha}{t} : \tau$}
    \end{prooftree}
  \end{minipage}
  \hfill
  \begin{minipage}{0.55\textwidth}
    \begin{prooftree}
      \AxiomC{$\Theta\mid\Gamma \vdash t : \tau$}
      \AxiomC{$\alpha \cnt \tau \in \Gamma$}
      \RightLabel{\textsc{Goto}}
      \BinaryInfC{$\Theta\mid\Gamma \vdash \goto{t}{\alpha} : \tau'$}
    \end{prooftree}
  \end{minipage}
  \hfill
  \vspace{1em}

  \begin{minipage}{0.45\textwidth}
    \begin{prooftree}
      \AxiomC{$\mathbf{data}\ T\ \{K\ \Gamma^{\prime}, \dots \}\in\Theta$}
      \AxiomC{$ \Theta\mid\Gamma \vdash \sigma:\Gamma^{\prime}$}
      \RightLabel{\textsc{Constructor}}
      \BinaryInfC{$\Theta\mid\Gamma \vdash K\ \sigma : T$}
    \end{prooftree}
  \end{minipage}
  \hfill
  \begin{minipage}{0.45\textwidth}
    \begin{prooftree}
      \AxiomC{$\mathbf{data}\ T\{\overline{K_i\ \Gamma_i)}\}\in \Theta$}
      \AxiomC{$\overline{\Theta\mid\Gamma,\Gamma_i\vdash t_i : \tau}$}
      \RightLabel{\textsc{Case}}
      \BinaryInfC{$\Theta\mid\Gamma\vdash \mathbf{case}\ t\ \mathbf{of}\ \{\overline{K_i\ \Gamma_i\Rightarrow t_i}\} : \tau$}
    \end{prooftree}
  \end{minipage}
  \hfill
  \vspace{1em}

  \begin{minipage}{\textwidth}
    \begin{prooftree}
      \AxiomC{$\mathbf{codata}\ T\ \{D\ \Gamma^{\prime}:\tau,\dots \}\in \Theta$}
      \AxiomC{$\Theta\mid\Gamma\vdash \sigma:\Gamma^{\prime}$}
      \AxiomC{$\Theta\mid\Gamma \vdash t :T$}
      \RightLabel{\textsc{Destructor}}
      \TrinaryInfC{$\Theta\mid\Gamma \vdash t.D\ \sigma : \tau$}
    \end{prooftree}
  \end{minipage}
  \hfill
  \vspace{1em}

  \begin{minipage}{0.45\textwidth}
    \begin{prooftree}
      \AxiomC{$\mathbf{codata}\ T \{\overline{D_i\ \Gamma_i:\tau_i}\}\in \Theta$}
      \AxiomC{$\Theta\mid\Gamma,\Gamma_i \vdash t_i:\tau_i$}
      \RightLabel{\textsc{Cocase}}
      \BinaryInfC{$\Theta\mid\Gamma \vdash \mathbf{cocase} \{ \overline{D_i\ \Gamma_i} \Rightarrow t_i \} : T$}
    \end{prooftree}
  \end{minipage}
  \hfill
  \begin{minipage}{0.45\textwidth}
    \begin{prooftree}
      \AxiomC{$\mathbf{def}\ f\ \Gamma^{\prime} :\tau \in \Theta$}
      \AxiomC{$\Theta\mid\Gamma \vdash \sigma:\Gamma^{\prime} $}
      \RightLabel{\textsc{Call}}
      \BinaryInfC{$\Theta\mid\Gamma \vdash f\ \sigma : \tau$}
    \end{prooftree}
  \end{minipage}
  \hfill
  \vspace{1em}
\end{minipage}
\begin{minipage}{\textwidth}
  \vspace{1em}
  \judgementbox{Substitution Typing}{$\Theta\mid\Gamma\vdash\sigma:\Gamma^{\prime}$}

  \begin{minipage}{0.25\textwidth}
    \begin{prooftree}
      \AxiomC{\phantom{$\Theta\mid\Gamma \vdash \Gamma^{\prime}$}}
      \RightLabel{\textsc{Subst}$_1$}
      \UnaryInfC{$\Theta\mid\Gamma\vdash\nil:\nil$}
    \end{prooftree}
  \end{minipage}
  \hfill
  \begin{minipage}{0.35\textwidth}
    \begin{prooftree}
      \AxiomC{$\Theta\mid\Gamma\vdash\sigma:\Gamma^{\prime} \quad \Theta\mid\Gamma\vdash t:\tau$}
      \RightLabel{\textsc{Subst}$_2$}
      \UnaryInfC{$\Theta\mid\Gamma \vdash \sigma,t : \Gamma^{\prime},x:\tau$}
    \end{prooftree}
  \end{minipage}
  \hfill
  \begin{minipage}{0.35\textwidth}
    \begin{prooftree}
      \AxiomC{$\Theta\mid\Gamma \vdash \sigma:\Gamma^{\prime} \quad c\cnt\tau\in\Gamma$}
      \RightLabel{\textsc{Subst}$_3$}
      \UnaryInfC{$\Theta\mid\Gamma \vdash \sigma,c: \Gamma^{\prime}, c\cnt \tau$}
    \end{prooftree}
  \end{minipage}
  \hfill
  \vspace{1em}
\end{minipage}
\begin{minipage}{\textwidth}
  \vspace{1em}
  \judgementbox{Kinding Types}{$\Theta\vdash\tau:\kappa$}
  \begin{minipage}{0.2\textwidth}
    \begin{prooftree}
      \AxiomC{}
      \RightLabel{\textsc{PrimKind}}
      \UnaryInfC{$\Theta \vdash \tyint : \mathbf{prim}$}
    \end{prooftree}
  \end{minipage}
  \hfill
  \begin{minipage}{0.35\textwidth}
    \begin{prooftree}
      \AxiomC{$\mathbf{data}\ T\ \{\, \ldots \} \in \Theta$}
      \RightLabel{\textsc{DataKind}}
      \UnaryInfC{$\Theta \vdash T : \mathbf{data}$}
    \end{prooftree}
  \end{minipage}
  \hfill
  \begin{minipage}{0.35\textwidth}
    \begin{prooftree}
      \AxiomC{$\mathbf{codata}\ T\ \{\, \ldots \} \in \Theta$}
      \RightLabel{\textsc{CodataKind}}
      \UnaryInfC{$\Theta \vdash T : \mathbf{codata}$}
    \end{prooftree}
  \end{minipage}
  \vspace{1em}
\end{minipage}
\begin{minipage}{\textwidth}
  \vspace{1em}
  \judgementbox{Well-formed Programs}{$\vdash \wellformed{\Theta}$}
  \begin{minipage}{0.2\textwidth}
    \begin{prooftree}
      \AxiomC{}
      \RightLabel{\textsc{Wf-Empty}}
      \UnaryInfC{$\vdash \wellformed{\nil}$}
    \end{prooftree}
  \end{minipage}
  \hfill
  \begin{minipage}{0.75\textwidth}
    \begin{prooftree}
      \AxiomC{$\vdash \wellformed{\Theta}$}
      \AxiomC{$\overline{\Theta,\mathbf{codata}\ T \{\dots\}\vdash\goodctx{\Gamma_i}}$}
      \AxiomC{$\overline{ \Gamma_i \vdash \tau_i:\kappa}$}
      \RightLabel{\textsc{Wf-Codata}}
      \TrinaryInfC{$\vdash \wellformed{\Theta,\mathbf{codata}\ T\ \{ \overline{D_i\ \Gamma_i : \tau_i}\}}$}
    \end{prooftree}
  \end{minipage}
  \hfill
  \vspace{1em}

  \begin{minipage}{0.45\textwidth}
    \begin{prooftree}
      \AxiomC{$\vdash \wellformed{\Theta}$}
      \AxiomC{$\Theta, \mathbf{def}\ \text{f}\ \Gamma\coloneq t \mid\Gamma \vdash t:\tau$}
      \RightLabel{\textsc{Wf-\surfacelang{}}}
      \BinaryInfC{$\vdash \wellformed{\Theta,\mathbf{def}\ \text{f}\ \Gamma \coloneq t}$}
    \end{prooftree}
  \end{minipage}
  \hfill
  \begin{minipage}{0.45\textwidth}
    \begin{prooftree}
      \AxiomC{$\vdash \wellformed{\Theta}$}
      \AxiomC{$\overline{\Theta,\mathbf{data}\ T\{\dots\}\vdash\goodctx{\Gamma_i}}$}
      \RightLabel{\textsc{Wf-Data}}
      \BinaryInfC{$\vdash \wellformed{\Theta,\mathbf{data}\ T\ \{ \overline{K_i\ \Gamma_i} \}}$}
    \end{prooftree}
  \end{minipage}
  \hfill
  \vspace{1em}
\end{minipage}
\begin{minipage}{\textwidth}
  \vspace{1em}
  \judgementbox{Well-formed contexts}{$\Theta\vdash \goodctx{\Gamma}$}
  \begin{minipage}{0.25\textwidth}
    \begin{prooftree}
      \AxiomC{\phantom{$x\in\nil$}}
      \RightLabel{\textsc{Ctx}$_1$}
      \UnaryInfC{$\Theta\vdash\goodctx{\nil}$}
    \end{prooftree}
  \end{minipage}
  \hfill
  \vspace{1em}
  \begin{minipage}{0.35\textwidth}
    \begin{prooftree}
      \AxiomC{$\Theta\vdash\goodctx{\Gamma} \quad x\notin\Gamma \quad \Theta\vdash\tau:\kappa$}
      \RightLabel{\textsc{Ctx}$_2$}
      \UnaryInfC{$\Theta\vdash\goodctx{\Gamma,x:\tau}$}
    \end{prooftree}
  \end{minipage}
  \hfill
  \vspace{1em}
  \begin{minipage}{0.35\textwidth}
    \begin{prooftree}
      \AxiomC{$\Theta\vdash\goodctx{\Gamma} \quad \alpha\notin\Gamma \quad \Theta\vdash\tau:\kappa$}
      \RightLabel{\textsc{Ctx}$_3$}
      \UnaryInfC{$\Theta\vdash \goodctx{\Gamma,\alpha\cnt\tau}$}
    \end{prooftree}
  \end{minipage}
  \hfill
  \vspace{1em}
\end{minipage}
\caption{Typing rules for \surfacelang{}}
\label{fig:fun-rules}
\end{figure}
Typing terms in \surfacelang{} is split into five different judgement forms:
\begin{itemize}
  \item Typing Terms: $\Theta\mid\Gamma\vdash t:\tau$
  \item Substitution Typing: $\Theta\mid\Gamma\vdash\sigma:\Gamma^{\prime}$
  \item Kinding Types: $\Theta\mid\tau:\kappa$
  \item Well-formed programs: $\vdash \wellformed{\Theta}$
  \item Well-formed environments: $\Theta\vdash \goodctx{\Gamma}$
\end{itemize}
The rules for typing terms and kinding types are fairly standard.
Given a (well-formed) program $\Theta$ and a typing environment $\Gamma$, they assign types $\tau$ to terms $t$ and kinds $\kappa$ to types $\tau$, respectively.
Since we use typing contexts and substitutions for data and codata types, we have additional rules for typing $\sigma$ given a context $\Gamma^{\prime}$ (in a program $\Theta$ with context $\Gamma$).
These rules ensure that each term $t$ in $\sigma$ can be typechecked using the type $\tau$ assigned to its corresponding variable $x$ in $\Gamma$.
That is, given a data declaration of type $T$ with constructor $K$ that has defined environment $\Gamma^{\prime}$, a term $K\ \sigma$ has type $T$, if and only if $\sigma$ contains the same number of terms as $\Gamma$ contains variables, and for each $t$ in $\sigma$, the corresponding $x$ in $\Gamma$ has a type $\tau$ such that $\Theta\mid\Gamma\vdash t:\tau$.
The same goes for codata declarations and similarly for cases and cocases.
Here, the order of terms in both $\sigma$ and $\Gamma$ are important, as the order of constructor (or destructor) arguments defined in the declaration needs to be maintained.
\begin{example}
  Take the example program in \cref{ex:fun-syntax}. 
  In this case, we have the program $\Theta= \mathbf{data}\ \mathtt{Listint}\ \{\mathtt{Nil}\ \Gamma_1, \mathtt{Cons}\ \Gamma_2\}$, where $\Gamma_1=\nil$ and $\Gamma_2 = x:\mathtt{Int}, xs:\mathtt{ListInt}$.
  When we then want to type check the body of $\mathtt{tl}$, $\caseof{ls}{\mathtt{Nil}\Rightarrow\mathtt{Nil}, \mathtt{Cons}(x,xs)\Rightarrow xs}$ within the program $\Theta$, we have to ensure the contexts for the constructors $\mathtt{Nil}$ and $\mathtt{Cons}$ in the case-expression match $\Gamma_1$ and $\Gamma_2$ in the data declaration (which in this case works).\\
  On the other hand, if we have a term \lstinline{Cons(1,Nil)}, we have the substitution $\sigma = 1,\mathtt{Nil}$, and type checking this term amounts to checking $\Theta\mid\Gamma \vdash 1:\mathtt{Int}$ and $\Theta\mid\Gamma\vdash \mathtt{Nil}:\mathtt{ListInt}$.
\end{example}

The last two judgements ensure well-formedness for programs and contexts used in the other rules.
These are used to ensure everything in a program $\Theta$ can be type and kind checked.
What is not included in the rules for brevity is an extra occurrence check, where we make sure there are no duplicate type definitions for the same type name and no multiple declarations for the same constructor/destructor name.
The same check occurs for contexts $\Gamma$, where we check each variable and each covariable only occurs with a single type and not multiple times.

